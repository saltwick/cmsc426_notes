\documentclass[10pt,landscape]{article}
\usepackage{multicol}
\usepackage{calc}
\usepackage{ifthen}
\usepackage[landscape]{geometry}
\usepackage{hyperref}
\usepackage{amsmath}

% To make this come out properly in landscape mode, do one of the following
% 1.
%  pdflatex latexsheet.tex
%
% 2.
%  latex latexsheet.tex
%  dvips -P pdf  -t landscape latexsheet.dvi
%  ps2pdf latexsheet.ps


% If you're reading this, be prepared for confusion.  Making this was
% a learning experience for me, and it shows.  Much of the placement
% was hacked in; if you make it better, let me know...


% 2008-04
% Changed page margin code to use the geometry package. Also added code for
% conditional page margins, depending on paper size. Thanks to Uwe Ziegenhagen
% for the suggestions.

% 2006-08
% Made changes based on suggestions from Gene Cooperman. <gene at ccs.neu.edu>


% To Do:
% \listoffigures \listoftables
% \setcounter{secnumdepth}{0}


% This sets page margins to .5 inch if using letter paper, and to 1cm
% if using A4 paper. (This probably isn't strictly necessary.)
% If using another size paper, use default 1cm margins.
\ifthenelse{\lengthtest { \paperwidth = 11in}}
	{ \geometry{top=.2in,left=.2in,right=.2in,bottom=.2in} }
	{\ifthenelse{ \lengthtest{ \paperwidth = 297mm}}
		{\geometry{top=1cm,left=1cm,right=1cm,bottom=1cm} }
		{\geometry{top=1cm,left=1cm,right=1cm,bottom=1cm} }
	}

% Turn off header and footer
\pagestyle{empty}
 

% Redefine section commands to use less space
\makeatletter
\renewcommand{\section}{\@startsection{section}{1}{0mm}%
                                {-1ex plus -.5ex minus -.2ex}%
                                {0.5ex plus .2ex}%x
                                {\normalfont\large\bfseries}}
\renewcommand{\subsection}{\@startsection{subsection}{2}{0mm}%
                                {-1explus -.5ex minus -.2ex}%
                                {0.5ex plus .2ex}%
                                {\normalfont\normalsize\bfseries}}
\renewcommand{\subsubsection}{\@startsection{subsubsection}{3}{0mm}%
                                {-1ex plus -.5ex minus -.2ex}%
                                {1ex plus .2ex}%
                                {\normalfont\small\bfseries}}
\makeatother

% Define BibTeX command
\def\BibTeX{{\rm B\kern-.05em{\sc i\kern-.025em b}\kern-.08em
    T\kern-.1667em\lower.7ex\hbox{E}\kern-.125emX}}

% Don't print section numbers
\setcounter{secnumdepth}{0}


\setlength{\parindent}{0pt}
\setlength{\parskip}{0pt plus 0.5ex}


% -----------------------------------------------------------------------

\begin{document}

\raggedright
\footnotesize
\begin{multicols}{3}


% multicol parameters
% These lengths are set only within the two main columns
%\setlength{\columnseprule}{0.25pt}
\setlength{\premulticols}{1pt}
\setlength{\postmulticols}{1pt}
\setlength{\multicolsep}{1pt}
\setlength{\columnsep}{2pt}

\section{Convolution and Filters}
\[
\text{Gaussian Kernel} \implies \frac{1}{16}\begin{bmatrix} 1 & 2 & 1\\ 2&4&2\\1&2&1\end{bmatrix}
\]
\[
\text{Sobel Kernel} \implies 
G_x = \begin{bmatrix} 1 & 0 & -1\\ 2&0&-2\\1&0&-1\end{bmatrix},
G_y = \begin{bmatrix} 1 & 2 & 1\\ 0&0&0\\-1&-2&-1\end{bmatrix}
\]
\[
Sharpen \implies \begin{bmatrix} 0&-1&0 \\ -1&5&-1 \\ 0&5&0 \end{bmatrix}
\]

\begin{itemize}
        \item Zero padding $\rightarrow$ Pad edges of matrix with all 0's
        \item Reflection padding $\rightarrow$ Reflect end of matrix 
        \item Repeated Cyclic $\rightarrow \begin{bmatrix} {\begin{vmatrix} y & z \end{vmatrix}} a & b &...&y&z \end{bmatrix}$
        \item Repeated Edge
        \item Box filters cause artifacts $\rightarrow$ Gaussian filters better for images
        \item Correlation is not associative $\implies$ cannot combine filters
\end{itemize}

\subsection{Projective Geometry}
\newlength{\MyLen}
\settowidth{\MyLen}{\texttt{letterpaper}/\texttt{a4paper} \ }
\begin{itemize}
        \item Pinhole Model
        \begin{itemize}
             \item Projection from 3D onto 2D plane
             \item Vanishing point at infinity $\implies$ line is in plane parallel to camera plane
            \[y = \frac{fY}{Z}, x = \frac{fX}{Z}\]
            \item Orthographic Projection keeps parallel lines
        \end{itemize}
\end{itemize}
Given 4 points on an image find the vanishing lines
$$
A = (2,3) , B = (5,6) , C = (10,15) , D = (11,17)
$$
Convert to 3D coordinates using $z= f$
$$
A = (2,3,1) , B = (5,6,1) , C = (10,15,1) , D = (11,17,1)
$$
Find $V_1$ by finding intersection of  $\overline{AB}$ and $\overline{BC}$
$\overline{AB} = \vec{a} \times \vec{b}$
$$
\begin{vmatrix} a_x & a_y & a_z \\ b_x & b_y & b_z \\ i&j&k\end{vmatrix} = i \cdot \underbrace{\begin{vmatrix} a_y & a_z \\ b_y & b_z\end{vmatrix}}_{\text{x}} - j \cdot \underbrace{\begin{vmatrix} a_x & a_z \\ b_x & b_z\end{vmatrix}}_{\text{y}} + k \cdot \underbrace{\begin{vmatrix} a_x & a_y \\ b_x & b_y\end{vmatrix}}_{\text{z}}
$$
Alternate Cross Product calculation
$$
c_x = a_y \cdot b_z - a_z \cdot b_y,
c_y = a_z \cdot b_x - a_x \cdot b_z, 
c_z = a_x \cdot b_y - a_y \cdot b_x
$$
Repeat cross product of $\vec{c}$ and $\vec{d}$ to get $\overline{CD}$
$$
\overline{AB} = (-3, 3, -3), \overline{CD} = (-,-,-)
$$
Cross Product of two lines $\rightarrow$ Vanishing Point

\subsection{Geometric Transformations}
\settowidth{\MyLen}{\texttt{multicol} }
\[
\text{Projective/Homography (8DoF/4pts)} \implies 
\begin{bmatrix}h_{11}&h_{12}&h_{13}
\\
h_{21}&h_{22}&h_{23}
\\
h_{31}&h_{32}&h_{33}\end{bmatrix}
\]

Invariants -  Colinearity, Cross Ratios

\[
\text{Affine (6DoF/3pts)} \implies 
\begin{bmatrix}a_{11}&a_{12}&t_x\\a_{21}&a_{22}&t_y\\0&0&1\end{bmatrix}
\]
Invariants - Parallelism, Area Ratios, Length Ratios
\[
\text{Similarity (4DoF/2pts)} \implies 
\begin{bmatrix}sr_{11}&sr_{12}&t_x\\sr_{21}&sr_{22}&t_y\\0&0&1\end{bmatrix}
\]
Invariants -  Angles, Length Ratios
\[
\text{Euclidean/Translation (3DoF/2pts)} \implies 
\begin{bmatrix}r_{11}&r_{12}&t_x\\r_{21}&r_{22}&t_y\\0&0&1\end{bmatrix}
\]
Invariants -  Angles, Length Ratios
\begin{itemize}
        \item Solve for homography given 4 points
        \[
                \begin{bmatrix}\prime{x_1} \\ \prime{y_1}\end{bmatrix} = \begin{bmatrix} m_1 & m_2 \\ m_3 & m_4\end{bmatrix} + \begin{bmatrix} t_1 \\ t_2 \end{bmatrix}               
        \]
        \[
                \begin{bmatrix}x_1 & y_1 & 0 & 0 & 1 & 0 \\ 0&0& x_1 & y_1 & 0 & 0 \\ ...&...&...&...&...&...\end{bmatrix}
                \begin{bmatrix}m_1\\m_2\\m_3\\m_4\\t_1\\t_2\end{bmatrix} 
                = 
                \begin{bmatrix}\prime{x_1}\\ \prime{y_1} \\ \prime{x_2}\\ \prime{y_2} \\ . \\.\end{bmatrix}
        \]
\end{itemize}

\subsection{RANSAC}
\settowidth{\MyLen}{\texttt{.author.text.} }
RANSAC process: 
\begin{enumerate}
        \item Select n feature pairs at random, n = points required for transform
        \item Compute affine transform $T$ - n points $\implies$ solve system for $T$ 
        \item Computer inliers $\rightarrow$ distance < thresh
        \item Keep largest set of inliers
        \item Recompute $T$ using all inliers $\rightarrow$ find $T$ using Least Squares
\end{enumerate}
RANSAC iterations:
For each iteration, probability of outlier is
\[
   p = 1 - (1 - p_{outliers})^n, n = \text{points required}   
\]
After $i$ iterations, probability of at least 1 iteration having zero outliers:
\[
   1 - p^i = c, c = \text{required confidence}     
\]
Solve above for $i$ or find number of iterations $N$ with required probability $p$ and outlier ratio $e$, and points required $n$
\[
   N = \frac{log(1-p)}{log(1-(1-e)^n)}     
\]
\section{Image Features}
\settowidth{\MyLen}{\texttt{.begin.quotation.}}
Harris Corner Detector
\begin{enumerate}
        \item Compute image derivatives $I_x, I_y$ by convolving I with derivatives of Gaussian filter
        \item Form Harris Matrix $\implies H(x,y) = \sum_{u,v} g(u,v) \begin{bmatrix}I_x^2 & I_x I_y \\ I_x I_y & I_y^2\end{bmatrix}$
        \item Select points such that $K(x,y) = \frac{det(H)}{trace(H)}$ is above a threshold
        \item Perform non-maximal suppression
\end{enumerate}
Invariant means that the value of the descriptor shouldn't change when the image undergoes a change in appearance (brightness, rotation, etc.)
This is useful for matching features between different images of the same scene which may look slightly different.
SIFT Process:
\begin{enumerate} 
        \item Compute Corners
        \item Adaptive Non Maximal Suppression
        \item Compute feature descriptor at each point
        \begin{enumerate}
                \item Take 40 px window centered on point
                \item Downsample to 8x8
                \item Blur 
                \item Convert to 64x1 array
        \end{enumerate}
        \item Compute SSD for each pair of points $\rightarrow$ keep matches where ratio between first best and second best is low
        \item RANSAC to compute Homography
\end{enumerate}
ANMS $\rightarrow$ Ensure points are equally distributed across image. Pick regional maxima.

\subsection{Color Spaces}
\settowidth{\MyLen}{\texttt{.begin.quotation.}}
$L^*a^*b^* \rightarrow$ lightness, green - red, blue - yellow. Change in value = change in importance
\\ HSL $\rightarrow$ hue, saturation, lightness
\\ Why use different spaces? May be able to identify patterns more invariantly in some spaces vs. others


\subsection{Optical Flow}
Vector field of pixel motion between images \\
Pencil - All lines through a single point \\
Ego Motion - How the system itself is moving \\
Rotation $\implies$ Flow becomes tangential to concentric circles around axis of rotation \\
Rotation around horiz/vertical axis with velocity in that direction $\implies$ tangential to concentric hyperbolas that limit to opposite \\
Aperture Problem $\implies$ Only information of a line moving through a hole is motion perpendicular to line \\
$I(x,y,t) = I(x,y,t) + \frac{\partial{I}}{\partial{x}} \frac{dx}{dt} + \frac{\partial{I}}{\partial{y}} \frac{dy}{dt} + \frac{\partial{I}}{\partial{t}} \frac{dt}{dt} $

Low texture regions (*i.e. sky*)
\begin{itemize}
        \item Gradients have small magnitudes 
        \item Small $\lambda_1$, small $\lambda_2$
\end{itemize}
High texture regions (*i.e. ground*)
\begin{itemize}
        \item Gradients are different, large magnitudes
        \item Large $\lambda_1$, large $\lambda_2$
\end{itemize}
Equations:
Rotational Vector = $\begin{bmatrix} \alpha \\ \beta \\ \gamma \end{bmatrix}$    Translational vector = $\begin{bmatrix} u \\ v \\ w \end{bmatrix}$
$$
u(x,y) = \frac{-U + xW}{Z},
v(x,y) = \frac{-V + yW}{Z} 
$$

Flow will be along lines that pass through the camera center
Flow values increase as you move further away from the origin since they are linear with $x$ and $y$ \\

Rotational flow equations:
$$
\dot{x} = \alpha x y - \beta(1+x^2) + \gamma y,\text{  } 
\dot{y} = \alpha(1 + y^2) - \beta x y + \gamma x
$$
Sum of translational and rotational optical flow
$$
u = \frac{W}{Z} (x - \frac{U}{W}) + \alpha x y - \beta(1+x^2) + \gamma y 
$$
$$
v = \frac{W}{Z} (y - \frac{V}{W}) + \dot{y} = \alpha(1 + y^2) - \beta x y + \gamma x
$$
Find flow from vector field: \\
Only have W and $\gamma \rightarrow$  Moving straight and rotating around axis of translation \\
Derotate $\rightarrow$ Check if flow makes a pencil

\subsection{GMM}
Multiple Gaussians to cover more varying color distributions
Steps: 1. Initialize $\pi, \mu, \Sigma$ to random. 
2. Alternate between expectation (eval model,assign points to clusters) and Maximization step (eval best parameters to fit points)
3. Estimate posterior with calculated $\pi, \mu, \Sigma$

\end{multicols}
\end{document}
